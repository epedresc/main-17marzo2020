\documentclass[10pt]{article}
\usepackage[italian]{babel}
\usepackage{mathrsfs}
\usepackage{graphicx}

% This is a helpful package that puts math inside length specifications
\usepackage{calc}

% Layout: Puts the section titles on left side of page
\reversemarginpar

%% Use these lines for letter-sized paper
%\usepackage[paper=letterpaper,
%            %includefoot, % Uncomment to put page number above margin
%            marginparwidth=1.2in,     % Length of section titles
%            marginparsep=.05in,       % Space between titles and text
%            margin=1in,               % 1 inch margins
%            includemp]{geometry}

%% Use these lines for A4-sized paper
\usepackage[paper=a4paper,
            %includefoot, % Uncomment to put page number above margin
            marginparwidth=30.5mm,    % Length of section titles
            marginparsep=1.5mm,       % Space between titles and text
            margin=25mm,              % 25mm margins
            includemp]{geometry}

%% More layout: Get rid of indenting throughout entire document
\setlength{\parindent}{0in}

%% This gives us fun enumeration environments. compactitem will be nice.
\usepackage{paralist,enumerate}

%% Reference the last page in the page number
\usepackage{fancyhdr,lastpage}
\pagestyle{fancy}
%\pagestyle{empty}      % Uncomment this to get rid of page numbers
\fancyhf{}\renewcommand{\headrulewidth}{0pt}
\fancyfootoffset{\marginparsep+\marginparwidth}
\newlength{\footpageshift}
\setlength{\footpageshift}
          {0.5\textwidth+0.5\marginparsep+0.5\marginparwidth-2in}
\lfoot{\hspace{\footpageshift}%
       \parbox{4in}{\, \hfill %
                    \arabic{page} of \protect\pageref*{LastPage}
                    \hfill \,}}

% Finally, give us PDF bookmarks
\usepackage{color,hyperref}
\definecolor{darkblue}{rgb}{0.29,0.0,0.55}
\definecolor{darkred}{rgb}{0.55,0.0,0.2}
\hypersetup{colorlinks,breaklinks,
            linkcolor=darkred,urlcolor=darkred,
            anchorcolor=darkred,citecolor=darkred}

%%%%%%%%%%%%%%%%%%%%%%%% New Commands %%%%%%%%%%%%%%%%%%%%%%%%%%%%
%%\title{DICHIARAZIONE SOSTITUTIVA DI ATTO DI NOTORIETA'}
%%\author{(art. 47 D.P.R. 28.12.2000 n. 445)}


% The title (name) with a horizontal rule under it
%
% Usage: \makeheading{name}
%
% Place at top of document. It should be the first thing.

\newcommand{\makeheadingTop}%
        {\hspace*{-\marginparsep minus \marginparwidth}%
          \begin{minipage}[t]
            {\textwidth+\marginparwidth+\marginparsep}%
                \textcolor{darkred}{\huge \bfseries DICHIARAZIONE SOSTITUTIVA \\[0.2\baselineskip]DI ATTO DI NOTORIET\`A\\[0.2\baselineskip] {\centering\small (art. 47 D.P.R. 28.12.2000 n. 445)}}\\[-0.15\baselineskip]%
                 \rule{\columnwidth}{1pt}%
        \end{minipage}}

\newcommand{\makeheading}[1]%
        {\hspace*{-\marginparsep minus \marginparwidth}%
         \begin{minipage}[t]{\textwidth+\marginparwidth+\marginparsep}%
                \textcolor{darkred}{\huge \bfseries Curriculum vit\ae~et studiorum\\[0.2\baselineskip] #1}\\[-0.15\baselineskip]%
                 \rule{\columnwidth}{1pt}%
        \end{minipage}}

\newcommand{\makeheadingNew}%
        {\hspace*{-\marginparsep minus \marginparwidth}%
         \begin{minipage}[t]{\textwidth+\marginparwidth+\marginparsep}%
                \textcolor{darkred}{\huge \bfseries Attivit\`a di ricerca tecnologica}\\[-0.15\baselineskip]%
                 \rule{\columnwidth}{1pt}%
        \end{minipage}}

        
% The section headings
%
% Usage: \section{section name}
%
% Follow this section IMMEDIATELY with the first line of the section
% text. Do not put whitespace in between. That is, do this:
%
%       \section{My Information}
%       Here is my information.
%
% and NOT this:
%
%       \section{My Information}
%
%       Here is my information.
%
% Otherwise the top of the section header will not line up with the top
% of the section. Of course, using a single comment character (%) on
% empty lines allows for the function of the first example with the
% readability of the second example.
\renewcommand{\section}[2]%
        {\pagebreak[2]\vspace{1.3\baselineskip}%
         \phantomsection\addcontentsline{toc}{section}{#1}%
         \hspace{0in}%
         \marginpar{
         \raggedright \large\scshape #1}#2}

\renewcommand{\subsection}[2]%
        {\vspace{0.7\baselineskip}%
         \phantomsection\addcontentsline{toc}{subsection}{#1}%
         {\bfseries #1}\\[0.1\baselineskip]#2}

% To add some paragraph space between lines.
% This also tells LaTeX to preferably break a page on one of these gaps
% if there is a needed pagebreak nearby.
\newcommand{\blankline}{\quad\pagebreak[2]}

%
% NOTE: \rcollength is the width of the right column of the table 
%       (adjust it to your liking; default is 1.85in).
%
\newlength{\rcollength}\setlength{\rcollength}{1.85in}%

%
\newcommand{\arxiv}[1]{\href{http://arxiv.org/abs/#1}{\textsf{arXiv:#1}}}
\newcommand{\cvline}[2]{\textbf{#1} -- #2\\}
\renewcommand{\cite}[1]{[Ref.~(\ref{#1})]}
\newcommand{\twocite}[2]{[Refs.~(\ref{#1})-(\ref{#2})]}
\newcommand{\email}[1]{\href{mailto:#1}{#1}}

\newif\ifitaly
\italyfalse
%%%%%%%%%%%%%%%%%%%%%%%%% Begin CV Document %%%%%%%%%%%%%%%%%%%%%%%%%%%%
\begin{document}
\makeheadingTop
%% \maketitle


\bigskip
\begin{minipage}{0.65\textwidth}
\begingroup
\let\center\flushleft
\let\endcenter\endflushleft
\makeheading{Elena Pedreschi}
\endgroup
\end{minipage}
\begin{minipage}{0.3\textwidth}
%\flushright\includegraphics[width=3.5cm]{gianiPhotoCV}
\end{minipage}

\section{Informazioni di contatto}
%% \ifitaly
\begin{minipage}[l]{0.5\textwidth}
via Manfredo Bertini, 127b\\
55049 Viareggio, Italy\\
(+39) 339 1800528\\
\email{elena.pedreschi@pi.infn.it}
\end{minipage}
\\[1.01\baselineskip]
\rule{\columnwidth}{1pt}

%\section{Objective}
%A Ph.~D.~position in the field of High Energy Physics.%, with particular interest in Charge Lepton Flavour Volation (CLFV) physics.
%%at hadron colliders.

\section{Informazioni personali}
\cvline{Data di nascita}{Giugno 21, 1969}
\cvline{Luogo di nascita}{Viareggio (LU), Italia}
\cvline{Cittadinanza}{Italiana}
%\textbf{Pagina web} --{\href{http://www.pi.infn.it/~pezzullo/index.html#}{ http://www.pi.infn.it/~pezzullo/index.html}}

%%%%%%%%%%%%%%%%%%%%%%%%%%%%%%%%%%%%%%%%%%%%%%%%%%%%%%%%%%%%%%%%%%%%%%%%%%%%%%%%


\section{Outline}
\begin{itemize}
   \item Ingegnere Elettronico in servizio come Tecnologo III livello  presso la Sezione INFN di Pisa con esperienza di oltre 17 anni nella progettazione, realizzazione, test, installazione e manutenzione di dispositivi e sistemi elettronici applicati a rivelatori di particelle installati ai collisionatori di particelle o nello spazio sulla Stazione Spaziale Internazionale;
   \item Principali responsabilit\`a in progetti sviluppati all'interno di Collaborazioni Nazionali e Internazionali:
   \begin{itemize}
%   \item Responsabile della produzione e del commissioning della parte digitale del DAQ del calorimetro elettromagnetico dell'esperimento Mu2e a Fermilab;
   \item {\bf Mu2e a Fermilab:} Manager DOE-L3 del progetto della parte digitale del DAQ del calorimetro elettromagnetico 
    (responsabile del progetto delle schede, dello sviluppo del firmware, dei test dei prototipi, della produzione delle schede, 
   del commissioning e futura manutenzione del sistema a Fermilab);
%    \item Responsabile dello sviluppo del firmware del DAQ del calorimetro di Mu2e;
    \item  {\bf Na62 al CERN:}  Responsabile del progetto hardware della scheda di trigger e DAQ "TEL62" 
    (responsabile della produzione, validazione e commissioning del trigger e DAQ dell'esperimento); TDAQ Expert
    per la presa dati;
%   \item Coordinatore della produzione, validazione e commissioning del sistema di trigger e acquisizione dati dell'esperimento Na62;
%   \item TDAQ Expert per la presa dati dell'esperimento Na62;
%   \item Responsabilit\`a di progetti sviluppati da collaborazioni nazionali e  internazionali:
   \end{itemize}
   \item Partecipazione e Coordinamento di Programmi di Ricerca finanziati su base competitiva:
   \begin{itemize}
   \item Co-Leader del WP10 "Trasferimento Tecnologico" del Progetto Europeo NEWS H2020-MSCA-RISE-2016¯ (Grant Agreement N° 734303, finanziamento complessivo 1,566,000 euro);
   \item Co-Leader del WP7 "Trasferimento Tecnologico" del Progetto Europeo INTENSE H2020-MSCA-RISE-2018 (Grant Agreement  n. 822185, finanziamento complessivo 2,116,000 euro);
   \item Co-Leader del WP8 "Dissemination and Outreach" del Progetto Europeo INTENSE H2020-MSCA-ITN-2019 (Grant Agreement  n. 858199 attualmente in preparazione, finanziamento complessivo 2,615,000 euro);
   \item Proponente del Progetto di Ricerca \emph{Photon irradiation of the Digitizer board Version 2 for the Mu2e electromagnetic calorimeter} sottomesso al Centro di Ricerche HZDR a Dresda, Germania;
   \item Responsabile del Progetto di Ricerca \emph{HAMLET: High bAndwidth coMmerciaL digitizer for hostile EnvironmenT} approvato dalla commissione CNTT (Bando “Research 4 Innovation – 2020”, finanziamento complessivo 25,000 euro).
   \end{itemize}
\end{itemize}




\section{Posizione Lavorativa Attuale}
\cvline{Aprile 2019 - in corso}{Contratto ex art. 23 da Tecnologo di III Livello professionale presso la Sezione INFN di Pisa \emph{"Progettazione, realizzazione, messa in prova e gestione di sistemi di acquisizione dati e di trigger ad alta banda, bassa latenza, basati su FPGA, per esperimenti di fisica delle particelle anche installati in ambienti caratterizzati da elevati livelli di radiazioni e campi magnetici"} (vincitrice del concorso relativo al Bando n.PI-T3-20364);}



\section{Titoli di studio e abilitazioni}
\cvline{Febbraio 2017}{Dottorato di Ricerca in Ingegneria dell'Informazione (XXIX ciclo), conseguito presso l'Universit\`a di Pisa con giudizio 'Ottimo'. Titolo della Tesi: \emph{"Design and implementation of an integrated fully digital trigger and data acquisition system for high energy physics experiments"};}

\cvline{Ottobre 2002}{Laurea Magistrale in Ingegneria Elettronica (Ordinamento antecedente al D.M. 509/99), conseguita presso l'Universit\`a di Pisa. 
Titolo della Tesi: \emph{"Messa a punto di laboratorio di testing e collaudo di prototipo di linea di ritardo integrata"};}
  
\cvline{Aprile 2015}{Ammessa a sostenere l'esame colloquio per un posto di Primo Tecnologo di II Livello Professionale presso l'INFN (telegramma INFN, Protocollo n. 3399 del 30 Aprile 2015 Oggetto: Bando n. 16622/2014);}

\cvline{Luglio 2009}{Conseguimento del Giudizio di Idoneit\`a per la eventuale costituzione di rapporti di lavoro a tempo determinato di personale tecnologo di III livello professionale, per coloro che non abbiano stipulato contratti di lavoro a tempo determinato con l'INFN, per l'area tecnologica di attivit\`a elettronica e informatica (vedi deliberazione 11122/2009 del Consiglio Direttivo dell'INFN);}

\cvline{Ottobre 2002}{Abilitazione alla Professione di Ingegnere, conseguita presso l'Universit\`a di Pisa - Sessione 2-2002 dell'anno 2002;}

\section{Esperienza professionale}
\cvline{Gennaio 2019 - oggi}{Intensity Frontier Fellowship a Fermilab. Programma di ricerca: \emph{"Construction of the Mu2e electromagnetic calorimeter"} (vincitrice della selezione \emph{Intesity Frontiers Fellow - Senior and Postdoctoral Researchers} relativa all'anno 2018, finanziamento complessivo 36,600 dollari);}


\cvline{Aprile 2019 - oggi}{Contratto ex art. 23 da Tecnologo di III Livello presso la Sezione INFN di Pisa (vincitrice del concorso relativo al Bando n. PI-T3-20364);}

\cvline{Aprile 2017}{Assegno di Ricerca Senior (vincitrice del concorso relativo al Bando n. 18811/2017) presso la Sezione INFN di Pisa nell'ambito della ricerca tecnologica sul tema \emph{"Progettazione e test di un sistema di digitalizzazione dei segnali nell'ambiente ostile dell'esperimento Mu2e"};}

\cvline{Febbraio 2013}{Assegno di Ricerca Senior (vincitrice del concorso relativo al Bando n. 16489/2012 (Settore scientifico disciplinare FIS/01) presso l'Universit\`a di Pisa sul tema \emph{"Sviluppo, installazione e test di un sistema di trigger e acquisizione dati ad alta banda per l' esperimento NA62 al CERN"};}

\cvline{Dicembre 2011}{Contratto d'Opera ai sensi degli articoli 2222 e seguenti del codice civile da parte dell' INFN (vincitrice del concorso relativo al Bando 14582/2011) sul tema \emph{"Svolgimento di attivit\`a  di progettazione, realizzazione, integrazione e verifica di sistemi elettronici per trigger e acquisizione dati in esperimenti di fisica delle particelle elementari"};}

\cvline{Dicembre 2010}{Borsa di Studio annuale del Dipartimento di Fisica dell' Universit\`a di Siena (vincitrice del concorso relativo al Bando D.R.N.2462 /2009-2010) finalizzata allo svolgimento di attivit\`a di ricerca sul tema \emph{"Potenziamento del Silicon Vertex Trigger all' esperimento CDF, ed esportazione della tecnica delle memorie associative a CMS"};}

\cvline{Ottobre 2010}{Contratto di Prestazione Occasionale presso il Consorzio Pisa Ricerche finalizzata allo svolgimento di attivit\`a tecnologica sul tema \emph{"Svolgimento di attivit\`a di progettazione di schede elettroniche e programmazione di sistemi di acquisizione dati"} presso la Sezione di Pisa dell'INFN sotto la direzione del Prof. Flavio Costantini e del prof. Alberto Del Guerra;}

\cvline{Agosto 2008}{Assegno di Ricerca nell' ambito della ricerca tecnologica dell'INFN (vincitrice del concorso relativo al Bando n. 12565/2008) sul tema \emph{"Test di qualificazione spaziale ed integrazione nel rivelatore, dell' elettronica di acquisizione per il calorimetro e.m. per l'esperimento AMS-02"};}

\cvline{Febbraio 2008}{Contratto d'Opera ai sensi degli articoli 2222 e seguenti del codice civile da parte del Consiglio Direttivo dell'INFN sul tema \emph{"Progettazione e test di qualificazione spaziale dell'elettronica di acquisizione per il calorimetro elettromagnatico di AMS. ASI I 035 07 0"};}


\cvline{Gennaio 2006}{Assegno di Ricerca nell'ambito della ricerca tecnologica dell'INFN (vincitrice del concorso relativo al Bando n.11126/05 ) sul tema \emph{"Progettazione e test di qualificazione spaziale dell'elettronica di acquisizione per il calorimetro elettro-magnetico di AMS"};}

\cvline{Dicembre 2003}{Borsa di Studio biennale per Tecnologi dell'INFN (vincitrice del concorso relativo al Bando n. B.C.9728/03);}

\cvline{Ottobre 2003}{Contratto di Prestazione Occasionale presso il Consorzio Pisa Ricerche finalizzata allo svolgimento di attivit\`a di ricerca tecnologica presso la Sezione di Pisa dell'INFN;}

\cvline{Dicembre 2002}{Contratto di Collaborazione Coordinata e Continuativa presso il Dipartimento di Fisica dell'Universit\`a degli Studi di Pisa finalizzata allo svolgimento di attivit\`a di ricerca tecnologica presso la Sezione di Pisa dell'INFN.}

%%%%%%%%%%%%%%%%%%%%%%%%%%%%%%%%%%%%%%%%%%%%%%%%%%%%%%%%%%%%%%%%%%%%%%%%%%%%%%%%
\bigskip
\begin{minipage}{0.65\textwidth}
\begingroup
\let\center\flushleft
\let\endcenter\endflushleft
\makeheadingNew{}
\endgroup
\end{minipage}

\bigskip \bigskip

Nel corso di oltre 17 anni di attivit\`a di ricerca ho contribuito
ad un gran numero di iniziative presenti nella Sezione di Pisa dell'INFN. Di particolare
rilievo sono stati i miei contributi alla progettazione, test, produzione, installazione e messa in opera di sistemi elettronici per gli esperimenti NA62, TOTEM, AMS02 e
R\&D per CMS e ILC. Attualmente sto lavorando allo sviluppo dell'elettronica
di readout e acquisizione dati del calorimetro elettromagnetico di Mu2e.\\


%In dettaglio:\\

\subsection{Esperimento Mu2e a Fermilab}

Dal 2015 sono membro attivo della collaborazione internazionale Mu2e. 
L'esperimento ricerca la conversione coerente 
del muone in elettrone senza produzione di neutrini nel campo di un nucleo 
di alluminio ed attualmente  \`e in costruzione al Muon Campus di Fermilab.
L'osservazione di questo processo fisico avrebbe una enorme importanza
dato che dimostrerebbe l'esistenza di fisica al di l\`a del Modello Standard. 
Faccio parte del gruppo INFN responsabile della costruzione del calorimetro
elettromagnetico ed, in particolare, sono "Manager DOE-L3" del progetto e 
costruzione del sistema di digitalizzazione e acquisizione dati del rivelatore
realizzato a Pisa.  
Il sistema \`e composto da 140 schede dette "DIgitizer ReAdout Controller",
o pi\`u brevemente DIRAC, ciascuna delle quali
digitalizza 20 canali alla frequenza di 200 MHz, pre-processa 
e trasmette i dati tramite link ottici al sistema di acquisizione dati globale dell'esperimento. 
Questo progetto si \`e rivelato particolarmente difficile, sia per la complessi\`a 
del sistema, che per le condizioni ambientali estremamente ostili alle quali saranno esposte
le schede durante la presa dati: campo magnetico di elevata intensit\`a, elevati livelli di radiazione e di vuoto, 
e nessuna possibilit\`a di manutenzione per lunghi periodi. Come illustrato nel seguito, 
l'impatto sul disegno del sistema \`e stato estremamente rilevante. 
Nel seguito sono descritti i miei contributi al progetto:

%Le schede saranno installate all'interno di un criostato 
%nel quale sono presenti un livello di vuoto di 10$^{-4}$ Torr
%ed un campo magnetico di elevata intensit\`a (1 T) 
%e saranno esposte ad elevato flussi
%& di radiazioni ionizzanti (TID 20-30 krad), neutroni O(10$^{11}$ n/cm$^{2}$ 1 MeV eq. Si) con p\textgreater 20 MeV. 
%Nel complesso questo ambiente \`e particolarmente ostile: il vuoto, il campo magnetico, la radiazione e l'impossibilit\`a di effettuare manutenzione rendono Mu2e un rivelatore operante in condizioni analoghe 
%a quelle di un esperimento nello spazio.
%Ultimo ma non meno importante fattore da tenere in considerazione \`e il costo che deve essere mantenuto
%ragionevolmente limitato.


\begin{itemize}
\item Definizione del sistema di acquisizione dati sulla base delle specifiche dell'esperimento;
\item Selezione dei componenti adatti ad operare nell' ambiente ostile di Mu2e. Questo ha richiesto una prima campagna di test di qualifica per dose ionizzante (TID), flusso di neutroni e funzionamento in campo magnetico;
\item Studio della distribuzione dell'albero di clock;
\item Studio della distribuzione delle alimentazioni con lo scopo di limitare la potenza dissipata;
\item Progettazione del primo prototipo della scheda DIRAC (Engineering Model);
\item Sviluppo del firmware in linguaggio VHDL per l'acquisizione di venti canali di ADC a 200MHz. Il firmware \`e stato sviluppato per il dispositivo Microsemi\textsuperscript{\textregistered}  SmartFusion2 SoC FPGA sotto l'ambiente Libero\textsuperscript{\textregistered};
\item Sviluppo del software  per la configurazione del Jitter Cleaner (LMK04828) e degli ADC (ADS4229) prodotti da Texas Instruments \textsuperscript{\textregistered}. La configurazione dei componenti viene effettuata tramite il processore ARM Cortex-M3 presente nella Smartfusion 2 SoC FPGA.  Il programma per il processore \`e stato scritto nel linguaggio C sotto l'ambiente SoftConsole di Microsemi\textsuperscript{\textregistered};
\item Per il progetto \emph{Photon irradiation of the digitizer board for the Mu2e electromagnetic calorimeter} presso la facility gELBE del Centro di Ricerche HZDR di Dresda, di cui sono stata co-proponente, sono stata responsabile dello sviluppo
dell'intero sistema di test. I test si sono svolti a Giugno 2018 ed hanno permesso di qualificare  i componenti fino a 30 krad; 
\item Lo stesso sistema di test \`e stato utilizzato presso la facility Calliope del Centro di Ricerche ENEA di Bracciano dove sono stati fatti irraggiamenti fino a 40 krad utilizzando una sorgente molto intensa di $^{60}$Co;
\item Sviluppo del sistema per il  "vertical slice test" del sistema completo di acquisizione dati: cristallo di Ioduro di Cesio (CsI), SiPM, elettronica di Front End e DIRAC;
\item Calibrazione del range dinamico e delle soglie di trigger;
\end{itemize}
 Nonostante i risultati soddisfacenti della prima campagna di irraggiamento, abbiamo deciso di apportare alcune modifiche alla DIRAC per renderla  pi\`u resistente all'esposizione alle radiazioni e campi magnetici e migliorarne le prestazioni:
 \begin{itemize}
 \item {FPGA}: abbiamo preferito utilizzare la FPGA Microchip\textsuperscript{\textregistered} PolarFire 
 per la maggiore immunit\`a delle celle di configurazione rispetto a problemi di single event upset.
 Inoltre, questa FPGA funziona fino a 700 MHz di frequenza; 
 \item {Transceiver Ottico}: abbiamo selezionato il VTRX, progettato al CERN per Atlas e CMS
  che ha anche il vantaggio del basso consumo;
 \item {DC-DC converter}: abbiamo selezionato LMZM33606 Texas Instruments \textsuperscript{\textregistered};
 \item Abbiamo inserito anche una interfaccia CANBUS e diversi sistemi di monitoring per  radiazioni (Radfet), correnti, tensioni e temperatura. 
\end{itemize}
\noindent
Per la nuova versione della DIRAC:  
\begin{itemize}
\item Ho supervisionato il lavoro di un ingegnere elettronico che ha modificato le schematiche della scheda;
\item Ho studiato la portabilit\`a del firmware dalle Microsemi\textsuperscript{\textregistered} Smartfusion2 FPGA SoC alle Microchip\textsuperscript{\textregistered}FPGA PolarFire;
\item Ho instanziato un processore Cortex M1 nella PolarFire FPGA che svolger\`a le stesse fuzioni che svolgeva il Cortex M3 nella SmarFusion2 (configurazione del Jitter Cleaner e degli ADC);
\item Sono stata fra i proponenti del Progetto di Ricerca \emph{Photon irradiation of the Digitizer board Version 2 for the Mu2e electromagnetic calorimeter} sottomesso al Centro di Ricerche HZDR a Dresda-Germania. HZDR ha accettato 
il progetto che abbiamo svolto presso la facility gELBE@HZDR nel mese di Maggio 2019.
\end{itemize}

Ho inoltre lavorato ad un altro importante progetto del gruppo di Pisa, 
si tratta dello sviluppo della stazione automatizzata di test per il controllo di qualit\`a e la caratterizzazione 
dei circa $4000$ SiPM utilizzati nel calorimetro.
La stazione ha permesso di caratterizzare rapidamente i SiPM a diverse 
temperature (da -10$^{\circ}$C a 25$^{\circ}$C), in un livello di vuoto di 10$^{-1}$ Torr, e 
senza interventi significativi da parte di un operatore durante ogni ciclo di test
effettuato su gruppi di 25 SiPM per volta. 
Dopo un primo periodo di collaudo a Pisa, 
abbiamo eseguito l'istallazione a Fermilab in una camera pulita dove sono effettuati
i test di tutti i SiPM. Il lavoro si \`e concluso nella estate 2019. 
I miei contributi principali sono stati:
\begin{itemize}
\item Il progetto hardware di tutte le schede elettroniche; 
\item La realizzazione ed il montaggio di tutte le parti elettroniche e meccaniche.
\end{itemize}
Infine, nell'estate 2019 ho scritto il Capitolato Tecnico per le due gare relative alla produzione delle schede DIRAC per un importo complessivo di 240,000 euro. 

\noindent
Queste attivit\`a mi hanno dato l'occasione di sviluppare e coordinare il Progetto di Ricerca
"HAMLET: High bAndwidth coMmerciaL digitizer for hostile EnvironmenT"
per la realizzazione di una versione commerciale del Digitalizzatore di Mu2e,
finanziato dalla commissione  CNTT dell'INFN (Bando “Research 4 Innovation – 2020”, 
finanziamento complessivo 25,000 euro).

\noindent
Sono co-autrice di numerosi articoli tecnici dedicati al progetto del calorimetro elettromagnetico
di Mu2e pubblicati su riviste internazionali e faccio parte della lista degli autori della collaborazione 
Mu2e per tutte le pubblicazioni dedicate ai risultati di fisica.

\noident
Allego lettera di presentazione di Dr. Stefano Miscetti dei Laboratori Nazionali di Frascati,
Project Manager del calorimetro e Responsabile Nazionale INFN
della Collaborazione Mu2e.





\subsection{Esperimento Na62 al CERN}
Negli anni 2009-2016 la mia principale attivit\`a si \`e svolta nell'ambito della collaborazione NA62, che si prefigge misure di precisione sui mesoni K all'acceleratore SPS al CERN.
Sono stata responsabile della progettazione, del test e della produzione della scheda TEL62 e del suo mezzanino TDCB che insieme costituiscono il sistema di trigger e acquisizione dati per buona parte dei rivelatori di Na62. 
Oltre 100 schede TEL62 e altrettante TDCB sono state installate nell'esperimento e sono state
utilizzate per la presa dati.
Le TEL62 ricevono dai TDCB i dati, digitalizzati, dei rivelatori e li memorizzano in memorie circolari in attesa di ricevere l'eventuale segnale di accettazione da parte del trigger. Parallelamente parte dei dati sono utilizzati per produrre le primitive di trigger.
La TEL62 \`e la scheda principale di tutto il sistema di acquisizione di NA62, ed \`e uno dei primi esempi di  sistemi integrati di uso generale per dati e trigger di tipo "fully digital". La stessa scheda, infatti, opportunamente configurata, pu\`o essere utilizzata per interfacciarsi a rivelatori anche molto diversi tra loro.
La produzione delle schede \`e terminata con successo nel settembre 2014 e l' installazione ha avuto luogo l'anno seguente. Attualmente NA62 \`e in presa dati.\\

\noindent
Per quanto riguarda la scheda TEL62:
\begin{itemize}
\item Sono stata responsabile del progetto hardware. La scheda ha un fattore di forma 9U Eurocard standard, i componenti utilizzati pi\`u importanti sono le FPGA Altera\textsuperscript{\textregistered} Stratix III EP3SL200F1152C4N (200K logic elements) e le memorie SDRAM DDR2 che implementano il buffer di memoria circolare;
\item Sono stata responsabile  della realizzazione, della produzione e del commissioning delle schede. Ho curato l'acquisto dei componenti, ho supervisionato l'attivit\`a del masterista che si \`e occupato dello sbroglio della scheda, della ditta che ha prodotto gli stampati e di quella che si \`e occupata del montaggio;
\item  Ho coordinato i test funzionali delle schede di pre-produzione e di produzione, per un totale di 115 schede; 
\item  Ho scritto le funzioni per effettuare il test Boundary Scan utilizzato per il debug delle schede;
\item Ho avuto la responsabilit\`a, condivisa con altri colleghi, di definire l'architettura del firmware delle FPGA;
\item Ho scritto in linguaggio vhdl, utilizzando l'ambiente di sviluppo Quartus II, il firmware per le Altera\textsuperscript{\textregistered} Stratix III  che gestisce il modulo  QDR  (Samsung K7Q161852AFC13);
\item Ho scritto in linguaggio vhdl, utilizzando l'ambiente di sviluppo Quartus II, il firmware per le Altera\textsuperscript{\textregistered} Stratix III che gestisce le memorie SDRAM DDR2 (Micron MT16HTF25664HZ-800H1);
\item ho valutato le performance del sistema una volta che \`e stato installato sull'esperimento e studiato le strategie per migliorararne le prestazioni (ottimizazione del firmware e studio dei costraint);
\end {itemize}
Le schede TDCB sono delle schede i cui componenti principali sono 4 High-Performance Time to Digital Converter (HPTDC) sviluppati al CERN. Ogni HPTDC ha 32 canali differenziali con una risoluzione temporale di 100ps.\\

\noindent
Per la scheda TDCB:
\begin {itemize}
\item Sono stata responsabile  della realizzazione, della produzione e del commissioning delle schede. Ho curato l'acquisto dei componenti, supervisionato l'attivit\`a del masterista che si \`e occupato dello sbroglio della scheda, della ditta che ha prodotto gli stampati e di quella che si \`e occupata del montaggio;
\item  Ho coordinato i test funzionali delle schede di pre-produzione e di produzione (oltre 100 schede);
\end {itemize}

Sono co-autore di numerosi articoli tecnici pubblicati su riviste internazionali e faccio parte della lista degli autori della collaborazione NA62 per tutte le pubblicazioni dedicate ai risultati di fisica.

\noident
Allego lettera di presentazione di Prof. Cristina Lazzeroni, professor in Particle Physics, University of Birmingham, UK - Na62 Spokesperson, CERN, Ch.

\subsection{Esperimento TOTEM}
TOTEM \`e uno degli esperimenti associati al collider LHC.
A partire dall' anno 2005 ho contribuito alla progettazione dell'elettronica associata al detector a GEM. 
L'installazione del sistema \`e avvenuta nel 2007.
In particolare mi sono occupata di:
\begin{itemize}
\item Progettare una scheda di conversione elettrica-ottica volta ad inviare i dati digitalizzati dal sito del detector alla control room per mezzo di fibre ottiche;
\item Selezionare tutti i componenti e i materiali idonei ad operare in ambiente fortemente radiaoattivo; 
\item Supervisionare lo sbroglio della scheda, la ditta che ha prodotto i circuiti stampati e quella che ha effettuato il montaggio delle schede;
\item Effettuare i test funzionali;
\item Effettuare l'installazione e il commissioning delle schede;
\item Nel 2013 ho avuto la responsabilit\`a di progettare l'upgrade dell' elettronica del sistema a GEM.
\end{itemize}
  
Sono co-autore di numerosi articoli tecnici su riviste internazionali ed ho fatto parte della lista allargata degli autori della collaborazione per le pubblicazioni dei risultati di fisica (fino anno 2013).

\subsection{Esperimento AMS-02}
A partire dal mese di Dicembre 2002 la mia principale attivit\`a lavorativa si \`e sviluppata nell'ambito dell'esperimento AMS-02. L'esperimento AMS-02 \`e collocato sulla Stazione Spaziale Internazionale dal Maggio 2011 ed \`e attualmente in fase di presa dati.
In particolare la mia attivit\`a \`e stata rivolta verso la progettazione, la produzione, il test e l'integrazione dell'elettronica del sistema di acquisizione dati e di trigger associati al Calorimetro Elettromagnetico.
L'attivit\`a di progettazione di sistemi elettronici operanti nello spazio richiede tecniche molto diverse da quelle convenzionali, dati i rigidi vincoli imposti dalle varie agenzie spaziali in termini di peso, consumo di potenza e soprattutto affidabilit\`a. 
Durante questi anni ho avuto la possibilit\`a di apprendere e mettere in pratica tali tecniche progettando tre tipi di schede che costituiscono il sistema di acquisizione dati e trigger del calorimetro, per un totale di circa 30 esemplari. Test di qualifica spaziale, svolti presso il Centro Interforze Studi e Applicazioni Militari (CISAM) di San Piero a Grado (Pisa) e il laboratorio INFN SERMS (Terni), nonch\`e un lungo test beam, svoltosi al CERN di Ginevra, hanno dimostrato la perfetta funzionalit\`a e affidabilit\`a del sistema e la sua idoneit\`a al volo e alla permanenza nello spazio.

In dettaglio, nell'ambito dell'esperimento AMS-02 ho avuto la responsabilit\`a della progettazione hardware delle seguenti schede:

\begin{itemize}
\item EDR (Ecal Data Reduction), \`e una scheda che ha la funzione di acquisire, memorizzare, processare e comprimere i dati digitalizzati provenienti dal Front End;
\item EPSFE (Ecal Power Supply Front End),  \`e una scheda che ha la funzione di gestire lo slow control del calorimetro nonch\`e di proteggere l' elettronica di Front End da possibili danni da radiazioni (latch-up); 
\item ETRG (Ecal Trigger), \`e una scheda che ha la funzione di processore di trigger di livello 0 e 1 associato al calorimetro;
\end{itemize}
Per ciascuno di questi progetti ho:
\begin{itemize}
\item definito gli schemi elettrici, utilizzando il CAD Orcad;
\item supervisionato il lavoro di un tecnico INFN incaricato dello sbroglio delle schede;
\item scritto per la scheda EDR, in linguaggio VHDL nell'ambiente di sviluppo Libero IDE\textsuperscript{\textregistered}, il codice per la FPGA Actel\textsuperscript{\textregistered}  (ACTA54SX32A-FPQG208) presente sulla scheda;
\item per ognuna delle schede sopra menzionate, al fine di valutarne la funzionalit\`a al variare di ogni possibile condizione operativa, ho progettato e coordinato la costruzione di alcune schede di test, atte ad inviare gli opportuni stimoli e registrare la risposta della scheda. 
\end{itemize}

Terminata la fase di sviluppo hardware ho coordinato la  preparazione e l'esecuzione dei test di qualifica per la verifica dell' idoneit\`a del sistema all'ambiente spaziale secondo le specifiche delle agenzie spaziali ASI, ESA e NASA.
In particolare: 

\begin{itemize}
\item	Test di resistenza alle radiazioni di vari componenti elettronici utilizzati nel sistema. Tali test si sono svolti presso il laboratorio GSI di Darmstadt; 
\item	Test di Emissione e Suscettibilit\`a elettromagnetica per verificare che il sistema rientri entro i limiti che la NASA impone ai sistemi che devono essere installati sulla ISS. I test sono stati effettuati presso 
l'installazione militare CISAM di San Piero a Grado (Pisa); 
\item	Test di resistenza alle vibrazioni, volti a validare l'analisi agli elementi finiti effettuata per il crate e verificare il processo di saldatura dei componenti sulle schede;
\item	Test in camera termica. Il sistema \`e stato sottoposto a 8 cicli termici, con variazioni di temperatura da -25$^{\circ}$C a +55$^{\circ}$C e viceversa, durante i quali \`e stato mantenuto costantemente sotto test.
\item	Test in camera di termovuoto. Il sistema \`e stato sottoposto nel vuoto a 4 cicli termici, con variazioni di temperatura da -25$^{\circ}$C a +55$^{\circ}$C e viceversa, durante i quali il sistema stato mantenuto costantemente sotto test. 
\item Test beam del calorimetro di AMS che si \`e svolto utilizzando la H4 Beam Line nella North Area del CERN Prevessin.
\end{itemize}

I test hanno dimostrato la perfetta funzionalit\`a del sistema e la sua idoneit\`a al volo e alla permanenza nello spazio.
Nel mese di agosto 2010 AMS-02 \'e stato portato al laboratorio NASA Kennedy Space Center in Florida e durante il mese di settembre 2010 ho seguito parte dei test di accettazione finale prima del volo avvenuto nel maggio del 2011.\\

Sono stato coautore di numerosi articoli tecnici su riviste internazionali ed ho fatto parte della lista degli autori della collaborazione per le pubblicazioni inerenti a risultati fisica (fino anno 2015).

\noident
Allego lettera di presentazione di Dr. Michael H. Capell
Senior Research Scientist, MIT, Cambridge, Massachusetts - AMS Avionics & Operations Lead.

\section{Attivit\`a tecnologiche varie}
Parallelamente alle precedenti attivit\`a ho poi contribuito ad altre iniziative.\

In particolare ho contribuito alla realizzazione dei seguenti progetti:

\subsection{R\&D per esperimento CMS}
Dal mese di dicembre 2010 ho partecipato ad una attivit\`a di R\&D legata ad un eventuale up-grade dell' esperimento CMS. Lo scopo era quello di dimostrare il possibile aumento di efficienza di trigger in caso di utilizzo di un sistema di memorie associative simile a quello in uso per l' esperimento CDF.
In particolare sono stata incaricata di:

\begin{itemize}
\item rimettere in funzione e testare il sistema di CDF;
\item sviluppare il firmware e il software necessari per poter caricare nel sistema di CDF le nuove banche di pattern modellate su una versione semplificata del tracciatore di CMS.
\end{itemize}

\subsection{R\&D ILC}
Agli inizi del 2005 si \`e formato nella Sezione di Pisa un gruppo di ricercatori e tecnologi che si proponevano di effettuare studi di R\&D relativi a tecnologie utilizzabili per l'International Linear Collider chiamato ILC.
Nell' ambito della collaborazione ILC ho partecipato alle seguenti attivit\`a:
\begin{itemize}
\item progettazione dell'elettronica di lettura per un sensore di posizione Wire Position Monitor (WPM) la cui funzione era misurare con grande precisione la posizione delle cavit\`a risonanti durante il raffreddamento. Nel mese di ottobre 2008 mi sono recata presso il laboratorio Desy ad Amburgo dove il primo prototipo di WPM \`e stato testato con successo;
\item R\&D di un sistema di controllo per il tuning delle cavit\`a risonanti. In particolare ho sviluppato una scheda di controllo digitale di piezo driver. Tale scheda interfacciata ad un processore ARM9 monta un doppio ADC, un doppio DAC bipolare e varie interfacce. La scheda in oggetto \`e stata utilizzata per studi di algoritmi di tuning di una cavit\`a risonante utilizzando un attuatore piezoelettrico. 
\end{itemize}
%%%%%%%%%%%%%%%%%%%%%%%%%%%%%%%%%%%%%%%%%%%%%%%%%%%%%%%%%%%%%%%%%%%%%%%%%%%%%%%%


\section{Esperienze presso Laboratori e/o Centri di Ricerca Nazionali e Internazionali}
Durante la mia esperienza lavorativa ho avuto modo di svolgere parte del mio lavoro presso i seguenti laboratori e centri di ricerca:
   \begin{itemize}
   \item CALLIOPE - Enea Casaccia, Roma, Italia
   \item CISAM - Pisa, Italia
   \item SERMS - Terni, Italia
   \item HZDR - Dresda, Germania
   \item CERN - Ginevra, Svizzera
   \item FERMILAB - Batavia, Illinois, USA
   \item GSI - Darmstadt, Germania
   \item INTA - Madrid, Spagna
   \item DESY - Amburgo, Germania
   \item ESTEC - Noordwijk, Olanda
   \item NASA Johnson Space Center - Houston, Texas, USA 
   \item NASA Kennedy Space Center - Cape Canaveral, Florida, USA
\end{itemize}
\section{Associazioni a Centri di Ricerca Nazionali e Internazionali}:
Sono associata a:
   \begin{itemize}
   \item \cvline{Novembre 2005 - oggi}{CERN-European Organization for Nuclear Research-Dipartimento PH Gruppo UFT-Ginevra-Svizzera ( CERN ID 97184).}
   \item \cvline{Dicembre 2003 - oggi} {INFN-Istituto Nazionale di Fisica Nucleare-Pisa-Italia.}
   \item \cvline{Novembre 2003 - oggi}{FERMILAB-Batavia-Illinois-USA (Fermilab ID 11355V).}
  \end{itemize}

%%%%%%%%%%%%%%%%%%%%%%%%%%%%%%%%%%%%%%%%%%%%%%%%%%%%%%%%%%%%%%%%%%%%%%%%%%%%%%%%

%\section{International Fellowships}
%\cvline{Anno 2018} {Vincitrice di una \emph{Intensity Frontier Fellowship} di Fermilab con un
%finanziamento complessivo di 36,000 dollari.\\
%Programma di ricerca: "Construction of the Mu2e electromagnetic calorimeter".}
%La Fellowship \`e iniziata a Gennaio 2019  ed \`e attualmente in corso.\\



%\cvline{Anno 2015} {Vincitrice di una \emph{URA VSP Fellowship}, 
%con un finanziamento complessivo di 6,000 dollari)\\
%Programma di ricerca: "Design, test and installation of the Data Acquisition System for the Mu2e calorimeter at Fermilab".}
%Durante la Fellowship ho sviluppato insieme ai colleghi di Fermilab l'architettura del sistema di digitalizzazione del %calorimetro di Mu2e ed ho definito le specifiche della scheda DIRAC.

\section{International Fellowships}
\cvline{2018} {\emph{Intensity Frontier Fellowship} di Fermilab (finanziamento 36,000 dollari).\\
Programma di ricerca: "Construction of the Mu2e electromagnetic calorimeter."}
La Fellowship \`e iniziata a Gennaio 2019  ed \`e attualmente in corso.\\

\cvline{2015} {\emph{Universities Research Association, INC (Washington, DC) - Visiting Scholars Program} (finanziamento 6,000 dollari).\\
Programma di ricerca: "Design, test and installation of the Data Acquisition System for the Mu2e calorimeter at Fermilab".}

\section{Corsi di formazione e seminari}
\cvline{Gennaio 2020}{\emph{Quantum technologies whitin INFN Status and perspectives} Padova - Università degli Studi, Palazzo del Bo'.}

\cvline{Novembre 2019}{\emph{Cadlog Technology Day: whorkshop sulla stampante 3D Dragonfly per la realizzazione di circuiti stampati} Milano - Cadlog - Via Luigi Beccali 4, Istituto Pirelli.}

\cvline{Novembre 2017}{\emph{Prospettive di applicazioni industriali sulla Tomografia Muonica} INAF - Osservatorio Astronomico di Padova Sala Jappelli.}

\cvline{Maggio 2016}{\emph{RF, Beam Instrumentation and Electrical Engineering for Particle Accelerator} Manfred Wendt, CERN (Switzerland), Dipartimento Ingegneria dell' Informazione dell'Universit\`a di Pisa, Pisa, Italia.}

\cvline{Luglio 2014}{\emph{Design technlogies for embedded multiprocessors system-on-chip} Rainer Leupers, RWTH Aachen University (Germany), Dipartimento Ingegneria dell'Informazione dell'Universit\`a di Pisa, Pisa, Italia.}

\cvline{Maggio 2014}{\emph{Accelerate System Performance with ALTERA SoC}, INFN, Pisa, Italia.}

\cvline{Maggio 2014}{\emph{Allegro High-Speed Constraint Management v16.3(System Interconnect Design - Allegro \& OrCAD, Advance with Engineer Explorer Series)}  di Cadence, INFN, Pisa, Italia.}

\cvline{Maggio 2013}{\emph{System On Chip} Altera, Bologna, Italia.} 

\cvline{Maggio 2013}{\emph{RF e Microonde} Agilent Technologies, Roma, Italia.}

\cvline{Febbraio 2013}{\emph{Misure EMI, Analisi di Segnali Complessi e Misure nel Dominio del Tempo ad Alta Risoluzione} Agilent Technologies, Polo Tecnologico di Navacchio, Cascina, Italia.} 

\cvline{Aprile 2010}{\emph{Uncovering Programmable Logic: From Trick to Solutions} Altera.} 

\cvline{Novembre 2008}{\emph{utilizzo degli FPGA con segnali High Speed} INFN, Bologna, Italia.}

\cvline{Ottobre 2007}{\emph{Integrit\`a dei segnali nelle moderne schede elettroniche} INFN, Bologna, Italia.}

\cvline{Settembre 2006}{\emph{Qualit\`a e progettazione di sistema per esperimenti di fisica nello spazio} INFN, Perugia, Italia.}

\cvline{Settembre 2004}{\emph{Componentistica e qualifica spaziale di dispositivi elettronici} INFN, Perugia, Italia.}


%%%%%%%%%%%%%%%%%%%%%%%%%%%%%%%%%%%%%%%%%%%%%%%%%%%%%%%%%%%%%%%%%%%%%%%%%%%%%%%%%


\section{Seminari e presentazioni a conferenze internazionali}
Sono autore di molti articoli tecnici, pubblicati su riviste peer-reviewed quali IEEE-TNS, NIM-A o JINST. Ho presentato la mia attivi\`a di ricerca tecnologica a molte conferenze internazionali ed in particolare IEEE Nuclear Science Symposium dove partecipo da diversi anni quasi sempre con una presentazione relativa alla mia attivit\`a.
Sono stata membro della lista degli autori di AMS-02, TOTEM e lo sono attualmente per NA62 e Mu2E per quanto riguarda le pubblicazioni relative a risultati di fisica.\\

Ho presentato la mià attività alle seguenti conferenze internazionali:\\

\cvline{Settembre 2019}{\emph{"The DIgitizer ReAdout Controller (DIRAC) of the Mu2e electromagnetic calorimeter at Fermilab"}, TWEPP-19, Santiago de Compostela, Spagna.}

\cvline{Novembre 2018}{\emph{"A DIgitizer ReAdout Controller for the MU2E calorimeter Data AcQuisition system"}, IEEE NSS- MIC, Sydney, Australia.}

\cvline{Novembre 2016}{\emph{"A fully digital trigger and data acquisition system for the NA62 kaon factory at the CERN SPS"}, IEEE NSS- MIC, Strasburg, France.}

\cvline{Maggio 2014}{\emph{"A high-resolution TDC-based board for a fully digital trigger and data acquisition system in the NA62 experiment at CERN"}, 19th IEEE REALTIME CONFERENCE, Nara, Giappone.}

\cvline{Settembre 2012}{\emph{"Firmware approach for TEL62 trigger and data acquisition board"}, TWEPP-12, Oxford, Inghilterra.}

\cvline{Ottobre 2011}{\emph{"TEL62 an integrated trigger and data acquisition board"}, IEEE NSS- MIC, Valenzia, Spagna.}

\cvline{Maggio 2010}{\emph{"Piezoelectric Actuators Control Unit"}, IPAC, Kyoto, Giappone.}

\cvline{Ottobre 2008}{\emph{"Readout and Control Electronics for the T2 Detector of the TOTEM Experiment"}, IEEE NSS- MIC, Dresda, Germania.}

\section{Collaboraz. Nazionali e Internazionali}

\cvline{2016 - 2019}{MUSE Project, H2020-MSCA-RISE-2015 seconded researcher for Work Package 2 \emph{Mu2e Detectors};}

\cvline{2016 - oggi}{membro della Collaborazione Internazionale Mu2e;}

\cvline{2010 - oggi}{membro della Collaborazione Internazionale Na62;}

\cvline{2006 - 2014}{membro della Collaborazione Internazionale TOTEM;}

\cvline{2005 - 2010}{membro della Collaborazione Internazionale ILC;}

\cvline{2002 - 2016}{membro della Collaborazione Internazionale AMS-02.}

\section{Attivit\`a di coordinamento e/o servizio}

\cvline{2018 - oggi}{Manager DOE-L3 della elettronica digitale del calorimetro elettromagnetico di Mu2e;}

\cvline{2017 - oggi }{Co-Leader WP10 per il Trasferimento Tecnologico nell'ambito del progetto Europeo NEWS H2020-MSCA-RISE-2016 Grant Agreement N. 734303;}

\cvline{2018 - oggi }{Co-Leader WP7 per il Trasferimento Tecnologico nell'ambito del progetto Europeo INTENSE H2020-MSCA-RISE-2016 Grant Agreement N. 822185;}

\cvline{2016 - oggi}{Svolgo attivit\`a di referaggio per la rivista internazionale \emph{Journal of Instrumentation}, in quanto sono considerata un esperto nel campo della conversione di segnali analogici in digitali per sistemi ad alta banda e bassa latenza;}
 
\cvline{2015 - oggi}{TDAQ Expert dell'esperimento Na62.} 

\section{Trasf. Tecnologico}
\cvline{2018}{Attivit\`a di trasferimento tecnologico nell'ambito dei progetti NEWS e INTENSE, tradotta nella collaborazione con la ditta francese Clever Operation riguardo lo sviluppo di un sistema di digitalizzazione commerciale, progettato e qualificato per utilizzo in ambiente ostile con alti livelli di radiazione;}

\noindent
Allego lettera di presentazione del Dr. SIA Radia, Founder and President CLEVER OPERATION Srl.



\section{Divulgazione Scientifica}
\cvline {Oggi}{Co-Leader del WP8 ”Dissemination and Outreach” del Progetto Europeo INTENSE H2020-MSCA-ITN-2019 (Grant Agreement n. 858199 attualmente in preparazione);}

\cvline{Luglio 2017 - oggi}{Collaboro alla organizzazione della Summer School della
Universit\`a di Pisa "Summer Students at Fermilab and other US Laboratories"  
e tengo seminari introduttivi per gli studenti sulle attivit\`a di ricerca tecnologica svolte 
all'Infn nel settore dell'elettronica digitale;}  

\cvline{Settembre 2015 - Settembre 2017}{Promozione dell'attivit\`a scientifica dell' INFN nell'ambito della Notte Europea dei Ricercatori (BRIGHT Toscana), per gli esperimenti Na62 e Mu2e. Nel corso delle edizioni dal 2015 al 2019 ho contribuito preparando diverse esperienze interattive per coinvolgere il pubblico delle visite guidate alla sezione di Pisa, utilizzando filmati, semplici setup sperimentali e poster;}  

\cvline{Marzo 2014 - Giugno 2014}{Ho partecipato all'organizzazione e all'installazione della mostra \emph{Balle di Scienza}, a cura dell'INFN, presso Palazzo Blu a Pisa. In particolare sono stata uno dei progettisti 
dell'elettronica associata a un piano inclinato dimostrativo che \`e stato accolto molto positivamente dal pubblico.} 



%%%%%%%%%%%%%%%%%%%%%%%%%%%%%%%%%%%%%%%%%%%%%%%%%%%%%%%%%%%%%%%%%%%%%%%%%%%%%%%%
\section{Lingue}
\cvline{Italiano}{lingua nativa}
\cvline{Inglese}{livello avanzato (QCER C1) in scrittura, lettura e conversazione}
\cvline{Francese}{livello elementare (QCER A2)in scrittura, lettura e conversazione}

%%%%%%%%%%%%%%%%%%%%%%%%%%%%%%%%%%%%%%%%%%%%%%%%%%%%%%%%%%%%%%%%%%%%%%%%%%%%%%%%
\section{Competenze informatiche}
\cvline{Sistemi operativi}{Windows, Unix/Linux}
\cvline{Linguaggi di programmazione}{C/C++}
\cvline{Linguaggi di descrizione hardware}{VHDL, Verilog}
\cvline{Strumenti CAD/CAE per la progettazione e verifica hardware}{Microsemi\textsuperscript{\textregistered} (Libero\textsuperscript{\textregistered} SoC, Libero\textsuperscript{\textregistered} SoC Polarfire) Altera\textsuperscript{\textregistered} (MAXPLUS\textsuperscript{\textregistered} II, QUARTUS\textsuperscript{\textregistered} II), Mentor Graphics\textsuperscript{\textregistered} (HDL Designer\textsuperscript{\textregistered}, ModelSim\textsuperscript{\textregistered}, JTAG Technologies (JTAG ProVision), Actel Libero IDE, Spice, Pspice, Cadence\textsuperscript{\textregistered}, Altium\textsuperscript{\textregistered} Designer, PROTEL99, OrCAD, CTA, LTSpice}
\cvline{Altro}{\LaTeX}


%%%%%%%%%%%%%%%%%%%%%%%%%%%%%%%%%%%%%%%%%%%%%%%%%%%%%%%%%%%%%%%%%%%%%%%%%%%%%%%%


%%%%%%%%%%%%%%%%%%%%%%%%%%%%%%%%%%%%%%%%%%%%%%%%%%%%%%%%%%%%%%%%%%%%%%%%%%%%%%%%


\section{Note legali}
La sottoscritta Elena Pedreschi  dichiara altres\`i di essere
informata, ai sensi e per gli effetti di cui all'art. 13 del Decreto
Legislativo 196/2003, che i dati personali raccolti saranno trattati,
anche con strumenti informatici, esclusivamente nell' ambito del
procedimento per il quale la presente dichiarazione viene resa.  \\ \\
\bigskip
\\
Pisa, \today

\noindent \hfill \hfill Il dichiarante

\end{document}

